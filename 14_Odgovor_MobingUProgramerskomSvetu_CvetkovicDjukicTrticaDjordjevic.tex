\documentclass[a4paper]{report}

\usepackage[T2A]{fontenc} % enable Cyrillic fonts
\usepackage[utf8x,utf8]{inputenc} % make weird characters work
\usepackage[serbian]{babel}
%\usepackage[english,serbianc]{babel}
\usepackage{amssymb}

\usepackage{color}
\usepackage{url}
\usepackage[unicode]{hyperref}
\hypersetup{colorlinks,citecolor=green,filecolor=green,linkcolor=blue,urlcolor=blue}

\newcommand{\odgovor}[1]{\textcolor{blue}{#1}}

\begin{document}

\title{Mobing u programerskom svetu\\ \small{Branko Cvetković, Tamara Đukić, Dusan Trtica, Petar Đorđević}}

\maketitle

\tableofcontents

\chapter{Recenzent \odgovor{--- ocena: 5/5} }

\section{O čemu rad govori?}
% Напишете један кратак пасус у којим ћете својим речима препричати суштину рада (и тиме показати да сте рад пажљиво прочитали и разумели). Обим од 200 до 400 карактера.
Rad objašnjava šta je to mobing, u kakvim uslovima se javlja i koje su mogućnosti zaštite od mobinga. Objašnjava koje sve vrste mobinga postoje, koje su faze mobinga i kakve posledice mobing može da ostavi po pojedinca i po firmu. Rad takođe navodi i na koji način je mobing regulisan u Srbiji i u svetu.

\section{Krupne primedbe i sugestije}
% Напишете своја запажања и конструктивне идеје шта у раду недостаје и шта би требало да се промени-измени-дода-одузме да би рад био квалитетнији.
Reference nisu adekvatno povezane. Klikom na redni broj koji predstavlja referencu ne dešava se ništa, a trebalo bi da odvede do odgovarajućeg dela u poglavlju Literatura. \odgovor{Popravljene su reference da vode na odgovarajuću stvar u literaturi.} U radu se koriste određene reči engleskog porekla, koje nisu naglašene kao engleske reči. Smatram da bi u tim slučajevima trebalo iskoristiti srpsku reč koja označava željeni pojam i u zagradi napisati koja je odgovarajuća engleska reč (na primer \textit{tim bilding} (eng. \textit{Team Building})). \odgovor{Dodati su prevodi engleskih reči.} Na samom kraju rada fali zaključak. \odgovor{Zaključak dodat.}

\section{Sitne primedbe}
% Напишете своја запажања на тему штампарских-стилских-језичких грешки
\begin{enumerate}
\item Sažetak, peta rečenica: Nema potrebe naglašavati da su ljudi žrtve mobinga (kao da još neko može da bude?). \odgovor{Izbačeno naglašavanje da su ljudi žrtve mobinga.}
\item Uvod: Na nekim mestima deluje kao da ima dupli razmak posle tačke. \odgovor{Ispravljeno.}
\item Uvod: Uvođenje engleskih termina, trebalo bi napisati i odgovarajuće srpske reči za te termine. \odgovor{Dodat i engleski termin.}
\item Poglavlje 1.3, prva stavka: Slovna greška u reči ''osoba''. \odgovor{Ispravljena slovna greška.}
\item Poglavlje 2, stavka 1.: Ispravno je ''pred-faza'' umesto ''pre-faza''. \odgovor{Ispravljeno u pred-fazu.}
\item Poglavlje 2, stavka 4.: Umesto '','' može ''i'' pošto nema daljeg nabrajanja. \odgovor{Prepravljeno.}
\item Poglavlje 3.2, pretposlednji pasus, poslednja rečenica: Slovna greška u reči ''konkrentnu''. \odgovor{Ispravljena slovna greška.}
\item Poglavlje 4: Slovna greška u reči ''potkrepljene''. \odgovor{Ispravljena slovna greška.}
\item Poglavlje 4.1, prvi pasus: Ponovo korišćenje engleskih reči i izraza, mislim da bi bilo ispravnije \textit{tim bilding} (eng. \textit{Team Building}) nego samo team building kao što je navedeno u tekstu. \odgovor{Prepravljeno.}
\item Poglavlje 4.1, predposlednji pasus: Engleski izraz \textit{skip line}, bilo bi lepo da se što približnije prevede, a da se u zagradi navede engleski izraz na koji se odnosi. \odgovor{Line prebačeno u level. Fraza nije prevedena jer se kod nas često koristi u ovom obliku i ne postoji ustaljeni prevod na srpski jezik.}
\item Poglavlje 4.1, poslednji pasus: Upotreba skraćenice \textit{HR}, iako je opšte poznata, trebalo bi navesti od čega je skraćenica jer se prvi put pojavljuje u radu. \odgovor{Dodato u listu akronima.}
\item Poglavlje 4.1, poslednji pasus: ''dj'' umesto ''đ'', takođe se pojavljuje na još nekoliko mesta u radu. \odgovor{Ispravljene slovne greške.}
\item Poglavlje 4.2, prvi pasus: Piše ''Top mendžment takođe treba imenuje'', a ispavnije bi bilo ''Top mendžment bi takođe trebalo da imenuje''. \odgovor{Prepravljeno.}
\item Poglavlje 4.2, prvi pasus: Pošto se \textit{(Leymann 1992b)} navodi kao referenca studija, to bi trebalo dodati u literaturu i odatle referisati. \odgovor{Dodata referenca.}
\item Poglavlje 4.2, prvi pasus: Ponovo engleski izrazi (\textit{flat} i \textit{middle management}). \odgovor{Flat ostavljeno kao ustaljeni izraz. Middle management prevedeno.}
\item Poglavlje 4.2, drugi pasus: Slovna greška u reči ''paradoksalno''. \odgovor{Ispravljena slovna greška.}
\item Poglavlje 4.2, drugi pasus: Piše ''tle'', ispravno je ''tlo''. \odgovor{Nije ispravljeno (\textit{Rečnik Matice srpske} nudi "tle" kao prihvatljiv oblik imenice "tlo").}
\item Poglavlje 4.2, treći pasus: Za navedene titule bi trebalo napisati odgovarajuće srpske nazive. \odgovor{Dodati srpski nazivi.}
\item Poglavlje 4.2, treći pasus: Prvo uvođenje skraćenice \textit{IT}, trebalo bi navesti od čega je to skraćenica. \odgovor{Dodato u listu akronima.}
\item Poglavlje 4.3: Slovna greška u reči ''stanju''. \odgovor{Ispravljena slovna greška.}
\item Poglavlje 4.4: ''dj'' umesto ''đ'' na nekoliko mesta. \odgovor{Ispravljena slovna greška.}
\item Poglavlje 4.4: Fali tačka na kraju poslednje rečenice. \odgovor{Ispravljena slovna greška.}
\item Poglavlje 5.1, drugi pasus: Trebalo bi staviti navodnike prilikom citiranja zakona. \odgovor{Ispravljena greška.}
\item Poglavlje 5.1, drugi pasus: Koristi se izraz ''mirljenju'', mislim da bi bilo adekvatnije reći ''pomirenju''. \odgovor{Prepravljeno.}
\item Poglavlje 5.1, treći pasus: Fali zatvorena zagrada nakon prve stavke. \odgovor{Ispravljena greška.}
\item Poglavlje 5.3: ''Prema podacima navedenim u [16]'', neadekvatan način navođenja reference. \odgovor{Ispravljena greška.}
\item Poglavlje 5.3: Tabela nije ispravno referencirana, trebalo bi da se citira. \odgovor{Prepravljeno cititanje na tabelu.}
\item Poglavlje 5.3: Ukoliko je moguće, trebalo bi da slika bude izoštrenija i bolje centrirana. \odgovor{Sliku nije moguće izoštriti jer je to njena originalna verzija. Centrirano tako da odgovara dužini reda.}
\item Poglavlje 5.3: Slika nigde nije referencirana. \odgovor{Dodata referenca.}
\end{enumerate}


\section{Provera sadržajnosti i forme seminarskog rada}
% Oдговорите на следећа питања --- уз сваки одговор дати и образложење

\begin{enumerate}

\item Da li rad dobro odgovara na zadatu temu?\\
Rad u velikoj meri odgovara na temu, ali bi mogao da se stavi veći fokus na mobing u samom programerskom svetu. \odgovor{Dodati su primeri iz programerskog sveta.}

\item Da li je nešto važno propušteno?\\
Pokrivena su sva pitanja zadata opisom teme, kao i neka druga koja
su takođe značajna za datu temu.

\item Da li ima suštinskih grešaka i propusta?\\
Ne, rad je uglavnom napisan u skladu sa opštim smernicama.

\item Da li je naslov rada dobro izabran?\\
Zbog prethodno navedene primedbe, smatram da naslov rada nije u skladu sa sadržajem rada. Potrebno je staviti veći akcenat na programerski svet. \odgovor{Dodati su primeri iz programerskog sveta.}

\item Da li sažetak sadrži prave podatke o radu?\\
Sažetak uglavnom sadrži prave podatke o radu. Jedino što odstupa od sadržaja rada, a navodi se u sažetku jesu primeri mobinga u svetu programiranja. \odgovor{Dodati su primeri iz programerskog sveta.}

\item Da li je rad lak-težak za čitanje?\\
Rad je relativno lak za čitanje. Svaka tema je lepo obrađena i redosled tema je adekvatno odabran.

\item Da li je za razumevanje teksta potrebno predznanje i u kolikoj meri?\\
Potrebno je predznanje na osnovnom nivou. U radu se koristi terminologija koje je uglavnom opšte poznata, iako bi neke termine trebalo dodatno objaniti prilikom uvođenja. \odgovor{Adekvatni prevodi su dodati.}

\item Da li je u radu navedena odgovarajuća literatura?\\
Literatura koje je navedena u radu je adekvatna. Takođe je i obimna, što je za pohvalu. Zamerka je način na koji je literatura navedena, jer nije praćena konvencija pisanja literature i nema odgovarajućih linkova. \odgovor{Ispravljeno.}

\item Da li su u radu reference korektno navedene?\\
Reference su u radu uglavnom adekvatno navedene, ali nisu adekvatno povezane (ne vode direktno do literature). \odgovor{Ispravljeno.}

\item Da li je struktura rada adekvatna?\\
Da, rad je adekvatno podeljen na smislene celine i njiv redosled je odgovarajujć, što u velikoj meri olakšva čitanje rada. Zamerka je nedostatak zaključka. \odgovor{Zaključak dodat.}

\item Da li rad sadrži sve elemente propisane uslovom seminarskog rada (slike, tabele, broj strana...)?\\
Da, rad sadrži sve elemente propisane uslovom seminarskog rada.

\item Da li su slike i tabele funkcionalne i adekvatne?\\
Slika i tabela jesu relevantne temi rada. Zamerka je što slika nije nigde referencirana. \odgovor{Ispravljeno, slika referencirana direktno iz teksta.}

\end{enumerate}

\section{Ocenite sebe}
% Napišite koliko ste upućeni u oblast koju recenzirate: 
% a) ekspert u datoj oblasti
% b) veoma upućeni u oblast
% c) srednje upućeni
% d) malo upućeni 
% e) skoro neupućeni
% f) potpuno neupućeni
% Obrazložite svoju odluku
U datu oblast sam srednje upućena. Kao neko ko je zaposlen, upućena sam u to šta je mobing i koje su mogućnosti zaštite od mobinga u kompaniji. U faze mobinga, kao i mobing u svetu sam veoma malo upućena.


\chapter{Recenzent \odgovor{--- ocena: 5/5} }


\section{O čemu rad govori?}
% Напишете један кратак пасус у којим ћете својим речима препричати суштину рада (и тиме показати да сте рад пажљиво прочитали и разумели). Обим од 200 до 400 карактера.
Rad govori o tome šta je mobing uopšte, kada nastaje, koji su tipovi i slično. Prolazi kroz Lajmanov i Egeov model mobinga. Ističe posledice mobinga i kako se protiv toga može boriti. Osvrće se na zakone koji daju podršku u borbi protiv mobinga, kako u Srbiji, tako i u svetu. Na kraju se analiziraju statistike rasprostranjenosti mobinga, kao i izloženosti ljudi mobingu.

\section{Krupne primedbe i sugestije}
% Напишете своја запажања и конструктивне идеје шта у раду недостаје и шта би требало да се промени-измени-дода-одузме да би рад био квалитетнији.
Čini mi se da je previše vremena posvećeno mobingu uopšte, pa možda ne bi bilo loše dodati još primera o mobingu u programerskom svetu. \odgovor{Primeri dodati.} Takođe, možda fali i neki zaključak da stavi tačku na celu priču. \odgovor{Zaključak dodat.}

Par referenci u literaturi mi nekako deluje nekompletno, na primer broj 5 - "Šta je mobing?", bez .bib datoteke ne može da se zaključi šta je to, kako naći taj izvor i slično. Nisam siguran da li se to tako radi, pa ne bi bilo loše uzeti ovo u razmatranje. \odgovor {Prepravljene su reference u literaturi. Dodati su linkovi za te stranice.}

%\odgovor{}
\section{Sitne primedbe}
% Напишете своја запажања на тему штампарских-стилских-језичких грешки
Na par mesta se koriste engleske reči, kao na primer u poglavlju 4.2 - middle management, što meni lično ne smeta, ali nisam siguran koliko je ispravno kada se piše rad na srpskom jeziku. \odgovor{Ispravljeno.}

Sadržaj i navođenje referenci u tekstu iz nekog razloga meni ne rade, odnosno na klik ne vode tamo gde bi trebalo. Takođe je citiranje u uvodu i 5. poglavlju navođeno nakon kraja rečenice, odnosno tačke, a mislim da bi trebalo da bude pre kraja. \odgovor{Popravljene su reference da vode na stavke u literaturi, stavljene su pre tačke.}

U poglavlju 4.4 drugi pasus je samo jedna rečenica, što mislim da nije pravilo kada se pišu radovi. Pretpostavljam da je ta rečenica trebalo da bude deo pasusa ispod, pa bi možda mogla da se samo spoji sa njim. Takođe je i u poglavlju 5 nakon naslova odmah podnaslov, što isto mislim da nije po nekim pravilima pisanja rada. \odgovor{Ispravljeno.}

U delu 1.2, kod stavke "Afektivni ili emotivni", u poslednjoj rečenici, deluje mi da je ispravnije da se kaže podmićivanjem svojih kolega na poslu, umesto svojim kolegama. \odgovor{Prepravljena greška.}

U poglavlju 1.3, kod prve tačke napisano je odoba umesto osoba. Takođe je i u poglavlju broj 2, pod 2, ukoliko treba da se koristi reč pojedinac, loše iskucana na dva mesta. \odgovor{Prepravljena greška.}

Kraj trećeg pasusa u delu 3.1, trebalo bi da piše prekomerna konzumacija, umesto prekomernu. Prva rečenica drugog pasusa u delu 3.2, u okruženju punom straha i stresa koju uzrokuje mobing - mislim da bi trebalo da glasi koje umesto koju. Takođe je i pred kraj pretposlednjeg pasusa napisano "konkurentsnu". \odgovor{Prepravljena greška.}

Poslednja reč u prvoj rečenici u 4. delu je loše iskucana. Dok je u drugoj rečenici u delu 4.2, top menadžment treba DA imenuje, zaboravljeno da. \odgovor{Prepravljena greška.} U ovom delu ima još par grešaka u kucanju, kao na primer "negatvino", "u programerskom kompanijama" umesto programerskim i "paradoskoksalno". \odgovor{Prepravljena greška.} Takođe, koristi se reč seniorniji, za koju nisam siguran da li postoji? \odgovor{Komparativ prideva \textit{senioran} izvedenog iz reči \textit{senior}, koja se često koristi u industriji za nekoga sa velikom količinom iskustva u industriji, ili nekog na višem nivou u hijerarhiji nadređenosti.}

Druga rečenica u delu 4.3, otkucana su dva veznika "i" jedan za drugim, kao i stanu umesto stanju. U poglavlju 5.1 fali jedna zatvorena zagrada, dok je u 5.3 u delu prve rečenice ispod slike, 'da je u tokom...', predlog 'u' višak. \odgovor{Prepravljena greška.}

Poslednje zapažanje je da fali l u gmail kod trećeg mejla. \odgovor{Prepravljena greška.}


\section{Provera sadržajnosti i forme seminarskog rada}
% Oдговорите на следећа питања --- уз сваки одговор дати и образложење

\begin{enumerate}
\item Da li rad dobro odgovara na zadatu temu?\\
Odgovara, mada kao što sam naveo u delu krupnih sugestija, deluje mi da bi bilo bolje posvetiti malo veći deo rada mobingu konkretno u programerskom svetu. \odgovor{Dodati su primeri koji se odnose na programiranje.}
\item Da li je nešto važno propušteno?\\
Mislim da nije.
\item Da li ima suštinskih grešaka i propusta?\\
Rekao bih da nema.
\item Da li je naslov rada dobro izabran?\\
Deluje mi da naslov nije ni menjan, a sa ovim naslovom deluje kao da je rad većinski skoncentrisan na programerski svet, što čini mi se nije slučaj. \odgovor{Dodati su primeri koji se odnose na programiranje.}
\item Da li sažetak sadrži prave podatke o radu?\\
Sadrži, osim što ne vidim da je rad završen primerima mobinga u svetu programiranja. \odgovor{Primeri dodati.}
\item Da li je rad lak-težak za čitanje?\\
Rad je relativno lak za čitanje.
\item Da li je za razumevanje teksta potrebno predznanje i u kolikoj meri?\\
Nije potrebno predznanje, ja nisam imao nikakvo predznanje a nadam se da sam dobro razumeo rad.
\item Da li je u radu navedena odgovarajuća literatura?\\
Jeste.
\item Da li su u radu reference korektno navedene?\\
Primedbe za reference sam naveo u delovima iznad, van toga bi trebalo da je dobro. \odgovor{Ispravljeno.}
\item Da li je struktura rada adekvatna?\\
Mislim da jeste, osim što možda fali zaključak. \odgovor{Zaključak dodat.}
\item Da li rad sadrži sve elemente propisane uslovom seminarskog rada (slike, tabele, broj strana...)?\\
Da.
\item Da li su slike i tabele funkcionalne i adekvatne?\\
Mislim da jesu.
\end{enumerate}

\section{Ocenite sebe}
% Napišite koliko ste upućeni u oblast koju recenzirate: 
% a) ekspert u datoj oblasti
% b) veoma upućeni u oblast
% c) srednje upućeni
% d) malo upućeni 
% e) skoro neupućeni
% f) potpuno neupućeni
% Obrazložite svoju odluku
Mislim da sam potpuno ili skoro neupućen, jer jedino što sam znao na ovu temu je šta otprilike znači reč mobing.

\chapter{Recenzent \odgovor{--- ocena: 4/5}}


\section{O čemu rad govori?}

Tema rada je Mobing u programerskom svetu i sastoji se od pet većih celina koje autori rada obrađuju. U prvom delu se detaljno definiše mobing i opisuje kada se jvlja, kako se ispoljava i koje vrste mobinga postoje. U drugoj celini je opisan hronološki tok događaja koji se javlja prilikom prve pojave mobinga, od nulte faze koja predstavlja uvertiru do poslednje i najekstremnije faze, isključenja sa posla. U sledećem odeljku su opisane posledice kako po pojedinca tako i po firmu u kojoj se pojavljuje mobing. U prvom delu ovog odeljka su opisane posledice koje zaposleni snosi na emotivnom i fizičkom nivou, dok je drugi deo posvećen uticaju mobinga na timski duh unutar firme i reputaciji. U pretposlednjem delu su opisane četiri veće celine koje su posvećene zaštiti od mobinga. U prvom delu su navedene preventivne mere koje je potrebno primeniti, da do mobinga ne bi došlo. U drugom delu je opisano na koji način i kako angažovanje menadžmenta utiče na sprečavanje mobinga. Treći deo je posvećen rehabilitaciji zaposlenog nakon pretrpljenog mobinga unutar kompanije, dok su u četvrtom delu navedeni pravni okviri zaštite od mobinga. U poslednjoj celini rada se navode postupci i regulative koje se primenjuju, kako u svetu tako i u Srbiji, da do mobinga ne bi došlo, sa osvrtom na neke sprovedene statistike.
\section{Krupne primedbe i sugestije}

Sam rad je korektno i adekvatno odrađen. Ispunjeni su svi uslovi koji su zahtevani pre pisanja rada. Pohvalio bih kolege na uloženom trudu i radu prilikom pisanja samog rada uz malu sugestiju za literaturu. Iako je sama literatura koju su koristili veoma obimna i sveobuhvatna, dodao bih da ima par propusta gde je naveden samo naslov knjige ili članka koji je korišćen. \odgovor{Popravljene su reference, dodati su linkovi.}
\section{Sitne primedbe}

\begin{enumerate}
    \item U podnaslovu 1.1 Kada se javlja mobing treba ili dodati ? na kraj ili promeniti podnaslov. \odgovor{Dodat upitnik.}
    \item Prva rečenica na drugoj stranici treba da piše psihičko ili fizičko, umesto ili psihičko ili fizičko. \odgovor{Popravljena greška.}
    \item Na drugoj stranici u odeljku Stresan posao treba da piše tenziji umesto tenzijama. \odgovor{Popravljena greška.}
    \item Na trećoj stranici u odeljku 1.2 Vrste mobinga treba da piše grupe [3] umesto grupe[3] i podmićivanjem svojih kolega umesto podmićivanjem svojim kolgama. \odgovor{Popravljena greška.}
    \item Na četvrtoj stranici na početku treba da piše kategorija [9] umesto kategorija[9] i kod Faza mobinga u delu 2. Prva faza treba da piše pojedinca umesto pojednica i pojedinca umesto pojednicna. \odgovor{Popravljena greška.}
    \item Na petoj stranici u delu Posledice po firmu u prvoj rečenici drugog pasusa treba da piše koji uzrokuje mobing umesto koju uzrokuje mobing. \odgovor{Popravljena greška.}
    \item Na šestoj stranici u delu preventivne mere treba da piše vanposlovne aktivnosti umesto van poslovne aktivnosti. \odgovor{Popravljena greška.}
    \item Na sedmoj stranici druga rečenica u odeljku Intervencija menadžmenta treba da piše menadžment umesto mendžment i posle treba nedostaje rečca da. \odgovor{Popravljena greška.}
    \item Na osmoj stranici u delu Profesionalna rehabilitacija u drugoj rečenici se veznik i uzastopno pojavljuje dva puta i treba da pi stanju umesto stanu. U sledećem odeljku teba da piše sprovođenje i unapređenje umesto sprovodjenje i unapredjenje. \odgovor{Popravljena greška.}
    \item Na devetoj stranici u prvoj rečenici treba da piše mirenju emsto mirljenju. U sledećem pasusu nedostaje zatvorena zagrada. \odgovor{Popravljena greška.}
    \item Na desetoj stranici treba da piše  [14,16] umesto [14][16]. \odgovor{Popravljena greška.}
\end{enumerate}

\section{Provera sadržajnosti i forme seminarskog rada}

\begin{enumerate}
\item Da li rad dobro odgovara na zadatu temu?\\
Rad je uspešno odgovorio na datu temu. Ima sve bitne aspekte koji su se u okviru date teme tražili.
\item Da li je nešto važno propušteno?\\
Ništa važno nije propušteno.
\item Da li ima suštinskih grešaka i propusta?\\
Nakon čitanja datog rada više puta, nije uočena nijedna suštinska greška niti propust.
\item Da li je naslov rada dobro izabran?\\
Jeste, naslov rada je korektno izabran.
\item Da li sažetak sadrži prave podatke o radu?\\ Da, ažetak sadrži prave podatke o radu.
\item Da li je rad lak-težak za čitanje?\\
Rad je veoma lagan za čitanje.
\item Da li je za razumevanje teksta potrebno predznanje i u kolikoj meri?\\
Nije potrebno skoro nikakvo predznanje za čitanje.
\item Da li je u radu navedena odgovarajuća literatura?\\
Da, u radu je navedena odgovarajuća literatura.
\item Da li su u radu reference korektno navedene?\\
Jesu, u radu su reference korektno navedene.
\item Da li je struktura rada adekvatna?\\
Struktura rada je adekvatna odgovarajućem šablonu.
\item Da li rad sadrži sve elemente propisane uslovom seminarskog rada (slike, tabele, broj strana...)?\\
Da, ovaj seminarski rad sadrži sve propisane elemente.
\item Da li su slike i tabele funkcionalne i adekvatne?\\
Da i slika i tabela su funkcionalne.
\end{enumerate}

\section{Ocenite sebe}
% Napišite koliko ste upućeni u oblast koju recenzirate: 
% a) ekspert u datoj oblasti
% b) veoma upućeni u oblast
% c) srednje upućeni
% d) malo upućeni 
% e) skoro neupućeni
% f) potpuno neupućeni
% Obrazložite svoju odluku
U datu oblast sam srednje upućen. Slušajući priče raznih zaposlenih i čitajući o informacijama koje se nalaze na internetu, stekao sam određeni nivo informisanosti o datoj oblasti.


\chapter{Dodatne izmene}
%Ovde navedite ukoliko ima izmena koje ste uradili a koje vam recenzenti nisu tražili. 

\end{document}
